\documentclass[12pt,a4paper,report]{ncc}

\usepackage{gazpromneft}

\begin{document}
\input{commands}

$\phantom{.}$ \vspace{2cm}

{\bf\LARGE\noindent Что можно делать на GitHub}

\renewcommand{\thesection}{\arabic{section}}
\def\ni{\noindent}

\section{О парадигме}

\ni При работе с GitHub-ом важно помнить идеологию, которая в него заложена:

\begin{enumerate}

	\item {\it Версия} любого документа представляет из себя последовательность {\it коммитов}. Каждый коммит — это изменение и перезапись {\bf всего файла целиком} (а точнее, некоторого непустого множества файлов — то есть, всего репозитория целиком).

	\item Версии называются {\it ветвями}. Каждый репозиторий может содержать в себе несколько ветвей, но среди них есть выделенная — {\tt master} — которая видна всем и является, в некотором роде, «парадной».

	\item Каждый человек работает в репозитории, который ему принадлежит — либо он создал этот репозиторий с нуля, либо его пригласили в репозиторий в качестве соавтора, либо он скопировал чужой репозиторий.

	\item Основная операция, посредством которой осуществляется работа GitHub — слияние двух ветвей. Если кто-то отредактировал у себя репозиторий, скопированный у другого человека (например, {\tt \href{https://github.com/fbakharev/gazpromneft-report}{/fbakharev/gazpromneft-report}}), он может запросить слияние своей ветви с общей, чтобы сделать изменения в оригинальном репозитории. Даже открытие встроенного в GitHub текстового редактора и правка кода онлайн — это создание новой ветви, а сохранение этого кода в {\tt master} — неявное слияние двух ветвей.

\end{enumerate}


\section{Как стать соавтором}

\ni Для этого нужно написать \underline{создателю} репозитория, он добавит нового пользователя в соавторы. Соавторство позволяет изменять файлы в {\tt master} без необходимости одобрять изменения у кого-либо ещё. Рекомендуется всем, принимающим активное участие в написании отчёта, стать соавторами.

\ms {\it Модераторами} будем называть создателя и соавторов.


\section{Что может делать соавтор}

\ni Главное — {\bfseries изменять файлы в {\tt master}}. Это растяжимое понятие включает в себя вообще всё, что только можно себе вообразить при работе с файлами: правка, создание, удаление, переименование, создание папок (главное — чтобы они сразу были непустыми).

\ms Большие возможности влекут большую ответственность: правки в {\tt master} невозможно отменить, и при внесении правок напрямую в {\tt master} система сайта не проверяет и не показывает, где конкретно были сделаны изменения. Поэтому рекомендуется вносить такие правки, только если они мизерны, либо над файлом работаете только вы, либо вы очень уверены в том, что нигде не совершили ошибок и не удалили важные изменения.

\ms Не менее главное — соавтор может создавать, обрабатывать и принимать запросы на слияние (т.н. {\it pull requests}). Создавать запросы на слияние могут вообще все, а вот их обработка и приём — дело исключительно соавторов.

\section{Если вы модератор: про запросы на слияние}

\ni Запрос на слияние создаётся тогда, когда автор изменений хочет убедиться (или уточнить у модераторов), что вносимые им изменения не нарушат работу проекта в целом. В частности, поэтому {\bf все} изменения не-соавторов проходят через запросы на слияние. Большинство pull request-ов система GitHub способна слить автоматически.

\ms Однако, \\
\rlap{$\phantom{.}$}\hspace{0.5in} (1) \label{var1} иногда GitHub неспособен выполнить слияние и {\it просит} модератора \\
\rlap{$\phantom{.}$}\hspace{0.5in} (2) \label{var2} модератор всегда, даже если GitHub готов сделать всё сам, {\it может} \\
внести изменения в ветвь, которую предлагают слить с {\tt master}.

\ms В первом случае GitHub предлагает сделать {\tt Resolve} {\tt conflicts}, где представляет документ целиком, указывая на конкретные конфликтные места, и модератор может сам написать, что конкретно с ними сделать. Возможно, удалить всё к чёрту и написать что-то совершенно другое, но сейчас слово модератора — закон\scolon и то, что выйдет из-под его пера, станет будущей версией вне зависимости от конфликтов с {\tt master}, которые могли там остаться).

\ms Во втором случае GitHub покажет, что он намерен удалить из ветви {\tt master}, а что добавить в неё из предлагаемой ветви при слиянии. Модератор может попросить открыть текстовый редактор, где перед ним предстанет будущая «слитая» версия в представлении GitHub. Он может внести в неё какие-то правки (опять же, сейчас его слово — закон): например, скопировать что-то, что GitHub хочет удалить, и впихнуть его в будущую версию.

\ms Практика показывает, что:

\def\lnka{\hyperref[var1]{(1)}}
\def\lnkb{\hyperref[var2]{(2)}}

\begin{enumerate}

\item Если в предлагаемой ветви одна строка текста заменена на другую, можно сделать только \lnkb. При автоматическом слиянии в коде останется только вторая строка.

\item Если в предлагаемой ветви большой кусок текста заменён на что-то, случается \lnka.

\item Если предлагаемая ветвь отличается от текущего состояния {\tt master} тем, что в одном месте убран кусок текста, а в другое место добавлен кусок текста, то GitHub предлагает автоматическое слияние (то есть \lnkb), и в «слитой» версии появятся {\bf оба} куска текста.

\end{enumerate}

\ms В частности, третий пункт удобен тем, что два человека могут независимо впечатывать свои куски текста в один файл {\bf оффлайн}, то есть, гарантируя успешную компиляцию файла, который потом появится в репозитории — и затем двумя последовательными запросами на слияние поместить в репозиторий плоды работы обоих.


\section{Что может делать произвольный пользователь GitHub}

\ni Вкратце: работать с запросами на слияние.

\begin{enumerate}
\setcounter{enumi}{-1}

\item Первым делом {\tt fork}: эта опция создаёт в подчинении данного пользователя копию текущей версии репозитория (как тот же самый {\tt \href{https://github.com/fbakharev/gazpromneft-report}{/fbakharev/gazpromneft-report}}).

\item Можно сделать pull request из оригинального репозитория в свой: это, разумеется, никак не изменяет общий репозиторий, а просто копирует его текущую версию (чтобы не отставать, если хочется вносить правки у себя).

\item Наконец, pull request из своего репозитория в общий: для помещения в текущую «отчётную» версию правок, внесённых в своей копии репозитория.

\end{enumerate}

Для опций „1.“ и „2.“ на страницах репозиториев есть кнопка „New pull request“, позволяющая выбрать, что и с чем хочется слить.

\begin{enumerate}
\setcounter{enumi}{2}

\item Делать в своём репозитории всё то же, что модератор может делать в общем: приглашать своих соавторов, обрабатывать запросы на слияние от каких-то других пользователей GitHub (например, своих прямых подчинённых).

\end{enumerate}

















\end{document}