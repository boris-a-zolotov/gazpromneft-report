\documentclass[12pt,a4paper,report]{ncc}

\usepackage{gazpromneft}

\begin{document}
\def\bea{\begin{eqnarray}}
\def\eea{\end{eqnarray}}

\def\beq{\begin{equation}}
\def\eeq{\end{equation}}

\def\beal{\begin{align*}}
\def\eeal{ \end{align*} }


\def\bfA{{\bf A}}
\def\bfB{{\bf B}}
\def\bfC{{\bf C}}
\def\bfD{{\bf D}}
\def\bfE{{\bf E}}
\def\bfF{{\bf F}}
\def\bfG{{\bf G}}
\def\bfH{{\bf H}}
\def\bfI{{\bf I}}
\def\bfJ{{\bf J}}
\def\bfK{{\bf K}}
\def\bfL{{\bf L}}
\def\bfM{{\bf M}}
\def\bfN{{\bf N}}
\def\bfO{{\bf O}}
\def\bfP{{\bf P}}
\def\bfQ{{\bf Q}}
\def\bfR{{\bf R}}
\def\bfS{{\bf S}}
\def\bfT{{\bf T}}
\def\bfU{{\bf U}}
\def\bfV{{\bf V}}
\def\bfW{{\bf W}}
\def\bfX{{\bf X}}
\def\bfY{{\bf Y}}
\def\bfZ{{\bf Z}}


\def\bbA{{\mathbb A}}
\def\bbB{{\mathbb B}}
\def\bbC{{\mathbb C}}
\def\bbD{{\mathbb D}}
\def\bbE{{\mathbb E}}
\def\bbF{{\mathbb F}}
\def\bbG{{\mathbb G}}
\def\bbH{{\mathbb H}}
\def\bbI{{\mathbb I}}
\def\bbJ{{\mathbb J}}
\def\bbK{{\mathbb K}}
\def\bbL{{\mathbb L}}
\def\bbM{{\mathbb M}}
\def\bbN{{\mathbb N}}
\def\bbO{{\mathbb O}}
\def\bbP{{\mathbb P}}
\def\bbQ{{\mathbb Q}}
\def\bbR{{\mathbb R}}
\def\bbS{{\mathbb S}}
\def\bbT{{\mathbb T}}
\def\bbU{{\mathbb U}}
\def\bbV{{\mathbb V}}
\def\bbW{{\mathbb W}}
\def\bbX{{\mathbb X}}
\def\bbY{{\mathbb Y}}
\def\bbZ{{\mathbb Z}}


\def\cA{{\mathcal A}}
\def\cB{{\mathcal B}}
\def\cC{{\mathcal C}}
\def\cD{{\mathcal D}}
\def\cE{{\mathcal E}}
\def\cF{{\mathcal F}}
\def\cG{{\mathcal G}}
\def\cH{{\mathcal H}}
\def\cI{{\mathcal I}}
\def\cJ{{\mathcal J}}
\def\cK{{\mathcal K}}
\def\cL{{\mathcal L}}
\def\cM{{\mathcal M}}
\def\cN{{\mathcal N}}
\def\cO{{\mathcal O}}
\def\cP{{\mathcal P}}
\def\cQ{{\mathcal Q}}
\def\cR{{\mathcal R}}
\def\cS{{\mathcal S}}
\def\cT{{\mathcal T}}
\def\cU{{\mathcal U}}
\def\cV{{\mathcal V}}
\def\cW{{\mathcal W}}
\def\cX{{\mathcal X}}
\def\cY{{\mathcal Y}}
\def\cZ{{\mathcal Z}}


\def\mfA{{\mathfrak A}}
\def\mfB{{\mathfrak B}}
\def\mfC{{\mathfrak C}}
\def\mfD{{\mathfrak D}}
\def\mfE{{\mathfrak E}}
\def\mfF{{\mathfrak F}}
\def\mfG{{\mathfrak G}}
\def\mfH{{\mathfrak H}}
\def\mfI{{\mathfrak I}}
\def\mfJ{{\mathfrak J}}
\def\mfK{{\mathfrak K}}
\def\mfL{{\mathfrak L}}
\def\mfM{{\mathfrak M}}
\def\mfN{{\mathfrak N}}
\def\mfO{{\mathfrak O}}
\def\mfP{{\mathfrak P}}
\def\mfQ{{\mathfrak Q}}
\def\mfR{{\mathfrak R}}
\def\mfS{{\mathfrak S}}
\def\mfT{{\mathfrak T}}
\def\mfU{{\mathfrak U}}
\def\mfV{{\mathfrak V}}
\def\mfW{{\mathfrak W}}
\def\mfX{{\mathfrak X}}
\def\mfY{{\mathfrak Y}}
\def\mfZ{{\mathfrak Z}}


\def\BB{{\mathcal B}}
\def\C{{\mathbb C}}
\def\E{{\mathbb E}}
\def\EE{{\mathcal E}}
\def\eps{{\varepsilon}}
\def\D{{\mathbb D}}
\def\F{{\mathcal F}}
\def\KK{{\mathcal K}}
\def\NN{{\mathcal N}}
\def\P{{\mathbb P}}
\def\PP{{\mathcal P}}
\def\R{{\mathbb R}}
\def\XX{{\mathcal X}}


%%%%%%%%%%%%%%%%
%%%%%%%%%%%%%%%%


\newcommand{\ve}{\varepsilon}
\newcommand{\vt}{\vartheta}
\newcommand{\vph}{\varphi}
\newcommand{\vfi}{\varphi}
\newcommand{\kp}{\kappa}

\newcommand{\mes}{\mathop{\rm meas}\nolimits}
\newcommand{\diag}{\mathop{\rm diag}\nolimits}
\renewcommand{\Re}{\mathop{\rm Re}\nolimits}
\newcommand{\diam}{\mathop{\rm diam}\nolimits}
\newcommand{\dist}{\mathop{\rm dist}\nolimits}
\newcommand{\id}{\mathop{\rm id}\nolimits}
\newcommand{\supp}{\mathop{\rm supp}\nolimits}
\newcommand{\capas}{\mathop{\rm cap}\nolimits}
\newcommand{\Endproof}{\hfill$\Box$}
\newcommand{\ora}{\overrightarrow}
\newcommand{\lip}{\mathop{\rm Lip}\nolimits}

\newcommand{\grad}{\mathop{\rm grad}\nolimits}
\newcommand{\df}{\partial}
\renewcommand{\div}{\mathop{\rm div}\nolimits}

\def\toalsur{\stackrel{\textrm{п.н.}}{\longrightarrow}}
\def\toprob{\stackrel{\P}{\to}}
\def\tomean{\stackrel{L_p}{\longrightarrow}}
\def\eqinlaw{\stackrel{d}{=}}

\def\Re{{\rm \,Re}}
\def\Im{{\rm \,Im}}
\def\kr{\widehat{\eta}}

\def\Weiss{{\rm We}}
\def\Peclet{{\rm Pe}}
\def\Rayleigh{{\rm Ra}}

\def\cov{{\rm cov}}
\def\ed#1{ {\mathbf 1}_{ \{#1  \}}}     
\def\Im{{\rm \,Im}}% indicator
\def\be#1\ee{\begin{equation}#1\end{equation}}


%%%%%%%%%%%%%%%%
%%%%%%%%%%%%%%%%


\usetikzlibrary{shapes, arrows, lindenmayersystems, decorations.fractals}

\tikzstyle{block} =
        [
                rectangle,
                draw,
                fill = blue!20,
                text width = 12.4em,
                text centered,
                rounded corners,
                minimum height = 4em
        ]


\tikzstyle{block1} =
        [
                ellipse,
                draw,
                fill = green!20,
                text width = 8em,
                text centered,
                rounded corners,
                minimum height = 4em
        ]

\tikzstyle{line} =
        [
                draw,
                -latex'
        ]

\tikzstyle{cloud} =
        [
                %draw,
                ellipse,
                fill = white!20,
                minimum height = 1em
        ]
        
\definecolor{my}{RGB}{102,204,0}

\DeclarePairedDelimiterX\Set[2]{\lbrace}{\rbrace}%
 { #1 \,\delimsize|\, #2 }


%%%%%%%%%%%%%%%%
%%%%%%%%%%%%%%%%


\def\r#1{\rotatebox{90}{#1}}

\def\skewcaption#1{%
	\stepcounter{figure}
	\caption*{\protect
		\r{\hspace{1.2cm}
			{\bf Рис.\,\thefigure.}
			#1
		}
	}
}

%%%%%%%%%%%%%%%%
% W --- для широких изображений
\def\AcustomFigW#1#2#3#4{
\vfill\eject{\begin{SCfigure}
		\centering
		\includegraphics[width=0.96\textwidth,natwidth=#1,natheight=#2]{Figures/#3}
		\skewcaption{#4}
	\end{SCfigure}
\clearpage}}
%
% H --- для высоких изображений
\def\AcustomFigH#1#2#3#4{
\vfill\eject{\begin{SCfigure}
		\centering
		\includegraphics[height=0.96\textheight,natwidth=#1,natheight=#2]{Figures/#3}
		\skewcaption{#4}
	\end{SCfigure}
\clearpage}}
%
%%%%%%%%%%%%%%%%


\newcounter{mdkvTabular}
\def\tabl#1{
	\stepcounter{mdkvTabular}
	{\bf
		Таблица
		\themdkvTabular:}
	#1\\
    \vspace{0.15cm}}


%%%%%%%%%%%%%%%%
% 1 --- ширина изображения
% 2 --- высота изображения
% 3 --- название файла
% 4 --- подпись
%
% W --- для широких изображений
\def\customFigW#1#2#3#4{
\vfill\eject{\begin{figure}
		\centering
		\includegraphics[width=0.96\textwidth,natwidth=#1,natheight=#2]{Figures/#3}
		\caption{#4}
	\end{figure}
\clearpage}}
%
% H --- для высоких изображений
\def\customFigH#1#2#3#4{
\vfill\eject{\begin{figure}
		\centering
		\includegraphics[height=0.96\textheight,natwidth=#1,natheight=#2]{Figures/#3}
		\caption{#4}
	\end{figure}
\clearpage}}
%
%%%%%%%%%%%%%%%%
%
% Подписи к изображениям
%
\def\caseno#1{Тестовый кейс №\,#1,\ }
%
\def\algname#1{алгоритм #1DCV,\ }
%
\def\restypePRES{давление}
\def\restypeDPX{ПХ исследуемой скважины}
\def\restypeINFL{влияние соседних скважин}
%
\def\graphtypeFirst{положение скважин и дебиты}
\def\graphtypeSecnd{ПХ и давление на исследуемой скважине}
%
%%%%%%%%%%%%%%%%
%
%\caseno{1}\algname{Proxy}\restypeINFL
%\caseno{1}\graphtypeSecnd
%
%%%%%%%%%%%%%%%%
%%%%%%%%%%%%%%%%


%%%%%%%%%%%%%%%%
%%%%%%%%%%%%%%%%
\definecolor{dop}{HTML}{91E6B4}
\definecolor{litl}{HTML}{FAFA8C}
\definecolor{mid}{HTML}{FADC5A}
\definecolor{big}{HTML}{EB8C32}
\definecolor{awf}{HTML}{F53255}
%
\def\dopcell#1{\cellcolor{dop} #1}
\def\litlcell#1{\cellcolor{litl} #1}
\def\midcell#1{\cellcolor{mid} #1}
\def\bigcell#1{\cellcolor{big} #1}
\def\awfcell#1{\cellcolor{awf} #1}
%%%%%%%%%%%%%%%%
%%%%%%%%%%%%%%%%


\newcommand{\footremember}[2]{%
    \footnote{#2}
    \newcounter{#1}
    \setcounter{#1}{\value{footnote}}%
}
\newcommand{\footrecall}[1]{%
    \footnotemark[\value{#1}]%
} 

$\phantom{.}$ \vspace{1.8cm}

{\bf\LARGE\noindent Что можно делать на GitHub}

\renewcommand{\thesection}{\arabic{section}}
\def\ni{\noindent}

\section{О парадигме}

\ni При работе с GitHub-ом важно помнить идеологию, которая в него заложена:

\begin{enumerate}

	\item Версия любого документа представляет из себя последовательность {\it коммитов \linebreak (commits)}. Каждый коммит~— это изменение и перезапись {\bf всего файла целиком} (а точнее, некоторого непустого множества файлов~— то есть, всего репозитория целиком).

	\item Версии называются {\it ветвями (branches)}. Каждый репозиторий может содержать в себе несколько ветвей, но среди них есть выделенная~— {\tt master}~— которая видна всем и является, в некотором роде, «парадной».

	\item Каждый человек работает в репозитории, который ему принадлежит~— либо он создал этот репозиторий с нуля, либо его пригласили в репозиторий в качестве {\it соавтора (collaborator)}, либо он скопировал чужой репозиторий.

	\item Основная операция, посредством которой осуществляется работа GitHub~— слияние двух ветвей {\it (merge или pull)}. Если кто-то отредактировал у себя репозиторий, скопированный у другого человека (например, {\tt \href{https://github.com/fbakharev/gazpromneft-report}{/fbakharev/gazpromneft-report}}), он может запросить слияние своей ветви с общей, чтобы сделать изменения в оригинальном репозитории. Даже открытие встроенного в GitHub текстового редактора и правка кода онлайн~— это создание новой ветви, а сохранение этого кода в {\tt master}~— неявное слияние двух ветвей.

\end{enumerate}


\section{Как стать соавтором}

\ni Для этого нужно написать \underline{создателю} репозитория, он добавит нового пользователя в соавторы. Соавторство позволяет изменять файлы в {\tt master} без необходимости одобрять изменения у кого-либо ещё. Рекомендуется всем, принимающим активное участие в написании отчёта, стать соавторами.

\ms {\it Модераторами} будем называть создателя и соавторов.

\vfill\eject
$\phantom{.}$ \vspace{-0.3cm}

\section{Что может делать соавтор}

\ni Главное~— {\bfseries изменять файлы в {\tt master}}. Это растяжимое понятие включает в себя вообще всё, что только можно себе вообразить при работе с файлами: правка, создание, удаление, переименование, создание папок (главное~— чтобы они сразу были непустыми).

\ms Большие возможности влекут большую ответственность: при внесении правок напрямую в {\tt master} система сайта не проверяет и не показывает, где конкретно были сделаны изменения~— отсюда можно не заметить возможной порчи файла. Поэтому рекомендуется вносить такие правки, только если они мизерны, либо над файлом работаете только вы, либо вы очень уверены в том, что нигде не совершили ошибок и не удалили важные изменения.

\ms Не менее главное~— соавтор может создавать, обрабатывать и принимать запросы на слияние {\it (pull requests)}. Создавать запросы на слияние могут вообще все, а вот их обработка и приём~— дело исключительно соавторов. Также запросы на слияние можно {\it закрывать (close pull request)}~— это способ отказа от ненужных изменений.

\section{Если вы модератор: про запросы на слияние}

\ni Запрос на слияние создаётся тогда, когда автор изменений хочет убедиться (или уточнить у модераторов), что вносимые им изменения не нарушат работу проекта в целом. В частности, поэтому {\bf все} изменения не-соавторов проходят через запросы на слияние. Большинство pull request-ов система GitHub способна слить автоматически.

\ms Однако, \\
\rlap{$\phantom{.}$}\hspace{0.5in} (1) \label{var1} иногда GitHub неспособен выполнить слияние и {\it просит} модератора \\
\rlap{$\phantom{.}$}\hspace{0.5in} (2) \label{var2} модератор всегда, даже если GitHub готов сделать всё сам, {\it может} \\
внести изменения в ветвь, которую предлагают слить с {\tt master}.

\ms В первом случае GitHub предлагает {\it разрешить конфликты (resolve conflicts)} и представляет документ целиком, две версии в одном тексте, указывая на конкретные конфликтные места~— модератор может сам указать, что конкретно с ними сделать. Возможно, удалить всё к чёрту и написать что-то совершенно другое, но сейчас слово модератора~— закон\scolon и то, что выйдет из-под его пера, станет будущей версией вне зависимости от конфликтов с {\tt master}, которые могли там остаться.

\vfill\eject

\ms Во втором случае GitHub покажет, что он намерен удалить из ветви {\tt master}, а что добавить в неё из предлагаемой ветви при слиянии. Модератор может попросить открыть текстовый редактор, где перед ним предстанет будущая «слитая» версия в представлении GitHub. Он может внести в неё какие-то правки (опять же, сейчас его слово~— закон): например, скопировать что-то, что GitHub хочет удалить, и впихнуть его в будущую версию.

\ms Практика показывает, что:

\def\lnka{\hyperref[var1]{(1)}}
\def\lnkb{\hyperref[var2]{(2)}}

\begin{enumerate}

\item Если в предлагаемой ветви одна строка текста заменена на другую, можно сделать только \lnkb. При автоматическом слиянии в коде останется только вторая строка.

\item Если в предлагаемой ветви большой кусок текста заменён на что-то, случается \lnka.

\item Если предлагаемая ветвь отличается от текущего состояния {\tt master} тем, что в одном месте убран кусок текста, а в другое место добавлен кусок текста, то GitHub предлагает автоматическое слияние (то есть \lnkb), и в «слитой» версии появятся {\bf оба} куска текста.

\end{enumerate}

\ms В частности, третий пункт удобен тем, что два человека могут независимо впечатывать свои куски текста в один файл {\bf оффлайн}, то есть, гарантируя успешную компиляцию файла, который потом появится в репозитории~— и затем двумя последовательными запросами на слияние поместить в репозиторий плоды работы обоих.


\vfill\eject
$\phantom{.}$ \vspace{-0.3cm}

\section{Что может делать произвольный пользователь GitHub}

\ni {\it Соучастник (contributor)}~— это человек, не являющийся соавтором данного репозитория, но участвующий в его работе и предлагающий изменения. Всё, что он может~— это работать посредством запросов на слияние.

\begin{enumerate}
\setcounter{enumi}{-1}

\item Первым делом {\tt fork}: эта опция создаёт в профиле данного пользователя копию текущей версии репозитория (как тот же самый {\tt \href{https://github.com/fbakharev/gazpromneft-report}{/fbakharev/gazpromneft-report}}).

\item Можно сделать pull request из оригинального репозитория в свой: это, разумеется, никак не изменяет общий репозиторий, а просто копирует его текущую версию (чтобы не отставать, если хочется вносить правки у себя).

\item Наконец, pull request из своего репозитория в общий: для помещения в текущую «отчётную» версию правок, внесённых в своей копии репозитория.

\end{enumerate}

Для опций „1.“ и „2.“ на страницах репозиториев есть кнопка „New pull request“, позволяющая выбрать, что и с чем хочется слить.

\begin{enumerate}
\setcounter{enumi}{2}

\item Делать в своём репозитории всё то же, что модератор может делать в общем: приглашать своих соавторов, обрабатывать запросы на слияние от каких-то других пользователей GitHub (например, своих прямых подчинённых).

\end{enumerate}

\end{document}